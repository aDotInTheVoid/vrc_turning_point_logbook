%!TeX root=../../../main.tex
\documentclass[../../../main.tex]{subfiles}

\begin{document}

\section{SOME DATE}

\subsection{Nixon: Drivetrain}

To start out with, Nixon build a drivetrain frame.
This is because we were not sure what a optimal claw would look like yet and Nixon hadn't build enough lift yet to be trusted with building one.
Therefor he was entrusted with the drivetrain as this was a common component that he had built before.
\par

A decision was made to build the drivetrain frame out of c-chanels at it had worked reasonably well in the past and we had plenty of them.
However the question now became how to properly join them at right angles into a strong and stable rectangle.
\par

On previous occasions we had decided just to screw the c-chanels together at the end with one screw, however this led to two problems.
\par

\begin{itemize}
	\item There was no garentee that the frame would stay square.
	      Sure, it might be held in place by by the other stuff we end up putting on but then again it might not.
	      And we want the drivetrain to suport the robot, and not the otherway round.
	\item Becase of the way they'd be atached (TODO: figure) the beems
	      runing perpendicular to eachother would we at slightly different
	      hights,
	      constraining what we could do in terms of building the rest of
	      the robot.
\end{itemize}

Therefor we needeed another, better way to atach the c-chanels together.

\subsection{Oscar: Claw}

A claw for rotating a object as large as a cap has never been built by anyone on our team before.
Therefor we decided to build physical prototypes of the claw to see what design would be better \par

We made two different prototypes which were quite similar in design, both were fixed beams coming from a plate at the back, which were simply to be driven into the caps and held.
It was decided that they should be held loosely so that there was room for human error when driven towards, and so that they would slide back off with ease once on the posts.
This decision was made becase past experiences had taught us the precious lession that fault tolerence was importand not only for driver speed but also for autonomous reliability, both of which yeilded lots of points.
\par

We had the standard Vex claw already in our kit, so we began by trying to use this, but quickly realised, after looking at the field specification, that it was not large enough and could not grab the central circle of the disk.
To combat this problem, we tried extending the arms of the claw to give it a larger reach but the claw was ineffective and we decided we could make a more specialised tool ourself.
\par

% TODO: Image

We made two different prototypes which were quite similar in design, both were fixed beams coming from a plate at the back, which were simply to be driven into the caps and held.
We spaced the beems further apart then was needed to fit which not only gave us the fault tolerence we discussed earlyer but also ment that they would slide back off with ease once on the posts.

The two designs were similar in function but the one that held the cap by the
central cylinder (as opposed to the outside rim) was chosen because it didn’t
require as much precision when lining up. This is due to the fact that the outside rim
is an octogon, so the radius varies around the shape. This means the cap will act differently
depending on where it is inserted to the rim design. However by griping the centeral circle,
a shaped defined by its constant radius, the interaction would always be the same, leading to
consistancy, which in turn creates reliability, which is the key to a good autonomous and
fast driven control.
\par

Then the design was rebuilt in aluminium and supports were added for additional structural strength.
Finally it was fixed to a backplate and motorised so it could rotate.

\end{document}
